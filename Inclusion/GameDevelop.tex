\chapter{Développement d'un jeu} % (fold)
\label{cha:développement_d_un_jeu}

Dans ce chapitre, je vais parler du développement du jeu <<Playboy-spots>>. <<Playboy-spots>> est le premier jeu que j'ai participé dans ma vie. Au début jusqu'au produit finale. Pendant ces processus de développement, j'ai appris beaucoup de chose, concept etc. Non seulement que j'ai appris beaucoup, mais aussi, grâce à ce projet, j'ai commencé à réfléchir beaucoup, comparer les techniques utilisés, résumer les points négative et positive. Ces expérience vont m'aider beaucoup dans les projets futurs. Dans n'importe quelle entreprise.

\section{Processus} % (fold)
\label{sec:processus}

Un jeu est un logiciel spéciale. Au niveau de ressource, un jeu est plus difficile à développer par rapport aux logiciel. Parce que un jeu demande beaucoup de ressource par rapport aux logiciels. Je vais le parler dans section suivant. Au niveau de processus de développement. Un jeu est plus simple par rapport aux logiciel. Ici, je vais parler le processus de développement d'un jeu en général et en comparant avec le processus réel chez chugulu games. surtout, il faut savoir, ici, le processus est utilisé par les petites équipes. Pour les grandes entreprises comme <<EA>>, ils ont un autre processus plus complet, plus systématiques, mais va durer plus long, et leurs processus est utilisé pour créer des jeux gros. Pour chugulu games et billions des studio comme chugulu games, leurs caractéristiques commune sont petite, bouger simple, réactiver rapide, développer rapide. 

\subsection{Processus Général} % (fold)
\label{sub:processus_général}

En générale, au niveau de programmation, le processus général n'est pas très loin que la processus classique d'un logiciel. Le processus peut être décomposé en plusieurs étape, qui sont présentés dans le schéma comme figure~\ref{fig:}.

% TODO add schéma générale

Dans ce figure, on peut voir, généralement, un développement du jeu peut être divisé en les étapes suivantes :

\subsubsection{Collect ideas} % (fold)

Développer un jeu peut avoir plusieurs raison. Peut être, nous avons trouvé des idées très intéressantes. Peut-être, nous avons trouvé que dans la marché du jeu, il manque ce genre de jeu, donc, le jeu ce que nous faisons va être réussis dans la futur. Peut-être, nous allons regarder un jeu qui est déjà réussis, qui a vendu billiard euros, et nous décidons de juste créer un jeu qui ressemble, mais en même temps, plus <<Intelligent>>. Peut-être, peut-être... Nous avons mille d'excuse pour créer un jeu. Pareil, selon les type de jeu, selon l'entreprise, nous pouvons avoir plusieurs processus totalement différent pour créer un jeu. 

Mais, dans toutes les processus différente, il existe un étape commune, essentiel et importante. C'est l'étape de collection des idées. 

Cet étape signifie à définir un principe du jeu en générale. Après cet étape, nous devons être sure que, notre jeu doit afficher quoi, comment jouer notre jeu. Donc, dans cet étape, ce que nous nous discutons est la chose la plus importante. 

Un autre point à savoir, c'est que, dans cet étape, souvent, nous savons pas les interfaces doivent afficher comment. Dans cet étape, souvent, une entreprise va organiser plusieurs réunions.

Dans le jeu de <<Playboy-Spots>>, dans cet étape, nous avons réussi a déterminer la coeur du jeu. Par exemple, nous savons que, nous avons plusieurs pack, chaque pack contient certaines photos. Nous savons que, sur chaque photo, il y aura certaine différences à trouver. Le nombre des différence n'est pas fixé, peut être 5, peut être 4. Nous savons que, nous avons 3 mode de jeu, sur chaque mode, le jeu va être joué différemment etc..

% subsubsection Collect ideas (end)

\subsubsection{Game Design} % (fold)

Dans cet étape, les game désigners vont commencer désigner le jeu. Concernant n'importe quel élément dans le jeu. Game désigners vont essayer de déterminer toutes les interfaces, les mécanismes (par exemple, les mécanisme de scoring, etc.) Les désigners vont essayer de sortir toutes les mock-up sur toutes les interfaces. 

En fait, cet étape est un début d'un cycle. Ce cycle commence par game désign, ensuite, les développeurs vont essayer de sortir une prototype, et à la fin, les développeurs vont montrer le prototype aux game désigners pour qu'ils puissent regarder ce qui donne, puissent savoir s'il y a des problème au niveau de conception, mécanisme, etc. Ce cycle peut durer longtemps, puisque, un jeu n'égale pas à un logiciel. Nous ne savons pas si nos désignes sont bonne ou pas, intéressantes ou pas. Donc il est essentiel de regarder notre idée devient réel, et jouer en réel. S'il y a des choses qui ne marchent pas, nous pouvons avoir de la chance de corriger avant commencer à programmer.

Normalement, après cet étape, nous devons être capable de sortie un cahier qui décrit toutes nos designs pour que nous puissions commencer développer.

% subsubsection Game Design (end) 

\subsubsection{Prototype} % (fold)

Dans les petite entreprise, ou les petits studios, cet étape est important. Souvent, dans ce cas, nous n'avons pas beaucoup de développeur, ni beaucoup de temps. Souvent, les jeux que nous allons développé ne sont pas gros. Ils sont les petite jeux. Il est possible de créer les prototypes dans ces cas. 

Un prototype est défini pas wikipédia comme suivante : Un prototype désigne le premier, ou l'un des premiers exemplaires d'un produit industriel (voiture, avion…). Cet exemplaire permet de faire des tests afin de valider les choix de conception de l'ensemble. Le prototype précède les exemplaires de pré-série. 

Dans le développement du jeu, notre prototype va déterminer le structure du jeu, les fonctionnalités générales dans le futur. 

% subsubsection Prototype (end)

\subsubsection{Work} % (fold)

Quand nous allons dans cette étape, nous somme sure que toutes les désigns sont déjà fait, validé par game désigner. Il n'existe pas grand problème dans prototype. Nous avons bien déterminer les éléments du jeu. Donc, nous pouvons commencer développer. 

Cette étape va duré longtemps, tous membres dans l'équipe vont rejoindre dans ce projet. En même temps, toutes les autre équipe vont commencer à travailler, par exemple l'équipe de graphiste, l'équipe de musique etc.. 

En fait, cette étape peut être divisé en plusieurs sous étapes. Prendre le programmation comme exemple, nous pouvons décomposé en plusieurs comme <<Faire la cahier de charge>>, <<Construire la structure général>>, etc..

% subsubsection Work (end)

\subsubsection{Beta} % (fold)

L'étape suivant est l'étape de Beta. Dans cette étape, normalement, nous avons bien développé notre jeu. Même s'il y a beaucoup des bugs, des erreurs, des fonctionnalités qui ne marchent pas. Le but de cette étape est de tester s'il y a gros problème à corriger. Non seulement les bugs qui font crasher le jeu, mais aussi les designes qui n'a pas l'air bien et utile. Dans ces 2 cas, il faut que nous collectons tous des informations nécessaires, trouvons des solutions, et donner aux développeurs ou désigners de soit corriger les bugs soit redésigner. 

Cette étape peut se répète plusieurs fois. En fait, aucun jeu peut être publié directement après développement. Ce n'est pas du tout possible. Il est normale d'avoir plain de bugs dans un jeu. Le jeu le plus complexe demande le plus de temps dans cette étape. Surtout le jeu en ligne comme <<World of Warcraft>>. Il avait fait 6 mois de beta public, c-à-d demande de joueur à jouer. Et avant ce commencer le beta public, il a déjà fait assez longtemps de beta test. Nous ne pouvons pas savoir précisément combien de temps a passé à l'interne de <<World of Warcraft>>, mais nous somme sur que cette étape est assez long. 

Il est pareil pour <<Playboy-Spots>>. Nous avons terminer le développement du jeu vers mi-aout, les reste du temps jusqu'à final, ne comporte que deux tâches. Une est débug, test le beta, corriger, une autre est amélioration. 

Amélioration est importante pour un jeu ou un logiciel. Souvent, il est pas si grave si un logiciel qui n'est pas amélioré. Sauf le logiciel très professionnel. En revanche, l'amélioration est souvent très grave pour un jeu. C'est à cause du complexité du jeu par rapport au logiciel. Et aussi le structure spécial du jeu. Un jeu peut prendre beaucoup de ressource du système, s'il n'est pas améliorer. C'est parce que un jeu comporte souvent un intelligent artifice complexe qui demande beaucoup de calcul. Un jeu comporte beaucoup des image, des son à afficher. Si ce jeu est un jeu 3D, il va prendre beaucoup plus de temps pour <<render>> des objets 3D sur l'écran. Souvent, un jeu est strictement besoin de fluide. 

L'amélioration peut comporte plusieurs aspects. Au niveau de mémoire, au niveau des ressource, au niveau de algorithmes, au niveau d'affichage, etc.. Les façons d'amélioration peut comporte beaucoup des façons différente aussi selon ce que nous allons améliorer. Elles peuvent être modifier les formats des images, remplacer les fonctions du système, utiliser multi-threading, utiliser multi-core, etc..

Dans <<Playboy-Spots>>, nous avons travaillé beaucoup sur améliorations. Y compris le fluide dans la boutique, diminue le temps de charge pour un jeu, diminue les mémoire consumé, etc.. 

% subsubsection Beta (end)



% subsection processus_général (end)

% section processus (end)

% chapter développement_d_un_jeu (end)