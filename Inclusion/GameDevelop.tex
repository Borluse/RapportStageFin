\chapter{Développement d'un jeu} % (fold)
\label{cha:développement_d_un_jeu}

Dans ce chapitre, je vais parler du développement du jeu <<Playboy-spots>>. <<Playboy-spots>> est le premier jeu que j'ai participé dans ma vie. Au début jusqu'au produit finale. Pendant ces processus de développement, j'ai appris beaucoup de chose, concept etc. Non seulement que j'ai appris beaucoup, mais aussi, grâce à ce projet, j'ai commencé à réfléchir beaucoup, comparer les techniques utilisés, résumer les points négative et positive. Ces expérience vont m'aider beaucoup dans les projets futurs. Dans n'importe quelle entreprise.

\section{Processus} % (fold)
\label{sec:processus}

Un jeu est un logiciel spéciale. Au niveau de ressource, un jeu est plus difficile à développer par rapport aux logiciel. Parce que un jeu demande beaucoup de ressource par rapport aux logiciels. Je vais le parler dans section suivant. Au niveau de processus de développement. Un jeu est plus simple par rapport aux logiciel. Ici, je vais parler le processus de développement d'un jeu en général et en comparant avec le processus réel chez chugulu games. surtout, il faut savoir, ici, le processus est utilisé par les petites équipes. Pour les grandes entreprises comme <<EA>>, ils ont un autre processus plus complet, plus systématiques, mais va durer plus long, et leurs processus est utilisé pour créer des jeux gros. Pour chugulu games et billions des studio comme chugulu games, leurs caractéristiques commune sont petite, bouger simple, réactiver rapide, développer rapide. 

\subsection{Processus Général} % (fold)
\label{sub:processus_général}

En générale, au niveau de programmation, le processus général n'est pas très loin que la processus classique d'un logiciel. Le processus peut être décomposé en plusieurs étape, qui sont présentés dans le schéma comme figure~\ref{fig:}.

% TODO add schéma générale



% subsection processus_général (end)

% section processus (end)

% chapter développement_d_un_jeu (end)