hn \chapter{Techniques informatiques} % (fold)
\label{cha:techniques_informatiques}

Dans ce chapitre, je vais parle des outils et des techniques utilisés pendant ce stage. Ces outils contient les outils utilisé dans l'équipe pour le but de mieux communiquer et mieux collaborer. Il y a des outils utilisé pour développement, y compris l'IDE, les outils pour débug et amélioration etc. Sauf-ci, je vais quand même présenter les frameworks, conception utilisé pour ce projet.

\section{Outils de Collaboration} % (fold)
\label{sec:outils_de_collaboration}

Pendant ce stage, j'ai touché plusieurs techniques nouvelles, appris beaucoup des façons à développement. Sauf ci, j'ai appris beaucoup des outils de collaborations, et apprendre la façon de travaille en équipe. Dans ce section, je vais parler des outils de collaborations

\subsection{Producteev et Redmine} % (fold)
\label{ssub:producteev_et_redmine}

La raison de mettre les 2 outils ensemble est non seulement les 2 outils se ressemblent, mais aussi, ils sont les plus importantes. Au début de stage, dans un première temps, ce que j'ai touché est le Producteev. Producteev est utilisé par Chugulu Games pendant certaines années. Cet été, ils ont décidé de changé à utiliser un autre outil qui s'appelle Redmine. 

\subsubsection{Producteev} % (fold)
\label{ssub:producteev}

% TODO Add the logo Producteev:

Producteev est un gestionnaire de tâches à la fois B2B et B2C dont l’ADN est collaboratif. D’une utilisation personnelle à un nombre de connectés illimités en entreprise, la solution fonctionne online et offline en interface avec les comptes e-mail, Facebook ou Google Calendar entre autres dans des versions web et mobiles. Il a ces fonctionnalités :
\begin{description}
	\item[Gestion de plusieurs équipes] Gérer plusieurs équipes grâce à Producteev en utilisant les fonctionnalités simples intégrées. Inviter les membres des équipes, ajustez les paramètres de confidentialité et bien plus encore pour votre liste de tâches.
	\item[Filtres et rapports] Générer facilement des rapports sur le projet et sur l'activité d'équipe. Par exemple, recevoir toutes les semaines un email d'aperçu des avancées d'équipe.
	\item[Emails récapitulatifs] Recevez chaque jour le rapport journalier/hebdomadaire, pour avoir un aperçu de l'équipe.
	\item[Tâches prioritaires]  L'algorithme utilisé pour la gestion des tâches prioritaires trie automatiquement les tâches et  suggère à faire ensuite
	\item[Synchroniser vos tâches avec Google Agenda] Synchronisez Producteev avec Google Agenda ou placer-y simplement une tâche à la fois. 
\end{description}

Sauf ces fonctionnalité, producteev est possible de synchroniser avec ses logiciel client sous Mac ou iPhone, avec Google. Et, il est possible de se communiquer en utilisant les mails. Envoyer et recevoir les nouvelles par les mails. 

% subsubsection producteev (end)

\subsubsection{Redmine} % (fold)
\label{ssub:redmine}

% TODO add the logo

Redmine est une application web Open Source de gestion complète de projet en mode web, développé en Ruby sur la base du framework Ruby on Rails.
Il a été créé par Jean-Philippe Lang. D'autres développeurs venant de la communauté des utilisateurs de Redmine contribuent depuis au projet. Il utilise la syntaxe Textile pour ses pages de wiki et de nombreux autres emplacements où l'utilisateur peut entrer du texte : 
\begin{itemize}
	\item descriptif de bogue ou de demande, ainsi que les commentaires associés
	\item forums
	\item nouvelles
	\item description
\end{itemize}

La fonctionnalité de Redmine comporte la liste suivante:
\begin{itemize}
	\item gestion multi-projets
	\item gestion fine des droits utilisateurs définis par des rôles
	\item gestion de groupes d'utilisateurs
	\item rapports de bogues (bugs), demandes d'évolutions
	\item Wiki multi-projets
	\item forums multi-projets
	\item news accessibles par RSS / ATOM
	\item notifications par courriel (mail)
	\item gestion de feuilles de route, GANTT, calendrier
	\item historique
	\item intégration avec divers suivis de versions : SVN, CVS, Mercurial, Git, Bazaar & Darcs
	\item identification possible via LDAP
	\item multilingue (25 langues disponibles pour la 0.7.0)
	\item support de plusieurs bases de données : MySQL, PostgreSQL ou SQLite
\end{itemize}

Par rapport au Producteev, redmine a des avantages evident. La plus importante est que redmine est gratuit. Il est possible d'installer redmine dans le site web interne d'entreprise. Il a un système de notification mieux que producteev. Redmine supporte beaucoup de personnalisation.

% subsubsection redmine (end)

% subsubsection producteev_et_redmine (end)

% section outils_de_collaboration (end)

\section{Outils de Développement} % (fold)
\label{sec:outils_de_développement}

% section outils_de_développement (end)

\section{Technique utilisé} % (fold)
\label{sec:technique_utilisé}

% section technique_utilisé (end)


% chapter techniques_informatiques (end)