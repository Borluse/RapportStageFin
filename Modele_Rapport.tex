\documentclass[twoside, UTF8]{EPURapport}
\input{include.tex}
\usepackage{picinpar}
\graphicspath{Image}



\thedocument{Rapporte de stage de fin d'étude}{Développement d'un jeu sur iOS}{Développement d'un jeu sur iOSz}

\grade{Département Informatique\\ 5\ieme{} année\\ 2011 - 2012}

\authors{%
	\category{Étudiant}{%
		\name{Fan ZHAO} \mail{zhao.fan@etu.univ-tours.fr}
	}
	\details{DI5 2010 - 2011}
}

\supervisors{%
	\category{Encadrant}{%
		\name{François Benaiteau} \mail{francois@chugulu.com}
	}
	\details{Chugulu Games}
}

\abstracts{Développement du jeu sur iOS. Ce rapport raconte mon stage de fin d'étude chez Chugulu Games. Y compris les concepts des développement du jeu, les techniques précisés etc.}
{iOS, jeux, développement}
{This repport is about the interneship at company Chugulu Games. The concepts about game develop, the process of game develop and the techniques used in game <<Playboy-Spots>>}
{iOS, game, develop}

\begin{document}

\chapter{Introduction} % (fold)

Le stage en 5ème année, qui est le stage de fin des étude, pour pouvoir obtenir le diplôme, est obligatoire pour tous les étudiants au département informatique. La durée est au minimum 4 mois, au plus 5 mois, soit 20 semaines, pendant l’été 2011. 

Le stage de fin des études est très important pour un étudiant. Parce que le durée de ce stage est 4 mois, qui est plus longue qu'avant. Aussi, après ce stage, nous allons diplômer. Nous devons chercher un boulot et commencer travailler. Nous serons plus étudiants. La durée de 4 mois nous permet d'avoir de la chance de participer dans un équpe. Travailler au dessus dans un grand projet. Pour ce durée, notre missions ne sont plus les petite. Cela nous permet de toucher des méchanismes d'une entreprises, des contrôle des projets, des nouvelles techniques, des applications au niveau d'entreprise, des façons au niveau d'entreprise pour développer, etc. Aussi, nous pouvons rencontrer les gens, apprendre les nouvelles idées etc. De l'autre coté, ce stage nous permet de savoir ce que nous en avons besoins, ce que nous aimer. Il est important puisque il y aura de la chance pour nous de continuer travailler après stage. Même si nous n'aurons pas de chance de continuer, il peut nous donner quelque idées pour nous aider dans la future de recherches des travails. Nous somme bénéficié aussi à partir de ce stage, est que nous pouvons observer une vraie entreprise. Cela peut nous aider pour créer notre propres entreprise dans la futur.

J’ai eu de la chance de trouver un stage dans l’entreprise Chugulu Games, qui est une entreprise à Tours, spécialisée dans le secteur des jeux. Mon stage a commencé du 23 Mai, jusqu'à le 30 Septembre. Je travail comme un développeur iOS. Ce stage est important pour moi, puisque travailler comme un développeur iOS est mon rêve. 

% chapter Introduction (end)

\chapter{Présentation} % (fold)

\subsection{Présentation de l'entreprise} % (fold)

\subsubsection{Chugulu Games} % (fold)

Chugulu Games est un studio de création spécialiste de la communication ludo-interactive et de l'advertainment. C'est concevoir et fabriquer des créations digitales (advergames,gaming banners, quiz interactifs, jeux multi-joueurs, serious games, social media applications) dont la dimension ludo-interactive, virale et communautaire est un accélérateur de notoriété et d'image de marque.

Voici le logo~\ref{fig:} de Chugulu Games.

Créée en 2006, Chugulu Games est une société spécialisée en advertainment. Nous concevons et fabriquons des créations digitales dont les dimensions ludique, virale et communautaire sont des accélérateurs de préférence de marque, de notoriété et de traffic.

Chugulu Games se concentre principalement sur 4 domaines principalement : Social Games, Jeux iPhone, Online Games et Unity 3D games. Comme le Figure~\ref{fig:} 

% TODO chugulu logos
Principalement dire, Chugulu games a 2 type de jeux suivant le techniques utilisés. Un type est le jeux sur internet, souvent développé par flash. Ce type de jeu contient le jeu <BlindTest> qui a un site web de chugulu, souvent, les autre jeux sont basés sur des sites web sociale. Par exemple, facebook. L'autre type principale est le jeux sur iPhone. Chugulu games commence a developper des jeux de 3D basé sur Unity 3d. 

% subsubsection Chugulu Games (end)

\subsubsection{Equipe} % (fold)
\label{sub:subsection_name}

Chgulu games a 2 équipe. Une équipe est à paris, qui est l'équipe principale, l'autre équipe est aux état-unis. Les 2 équipes sont présenté par le Figure~\ref{fig:} suivant:

% TODO 2 équipe de chugulu

L'équipe de paris se charge de développer les produits, l'équipe des état-unis se charge de la market des état-unis. 

% subsection subsection_name (end)

\subsubsection{Localisation} % (fold)
\label{ssub:subsubsection_name}

Les jeux de chugulu games tous ont multi-language. Normalement, les jeux de chugulu games ont des version d'anglais, version de français, version d'éspagnol. 

% subsubsection subsubsection_name (end)

\subsubsection{Références} % (fold)
\label{ssub:références}

Chugulu games a travaillé avec plusieurs entreprises. Ses partenaires comporte des plus grandes entreprises françaises. Chugulu games a crée des jeux d'avertissements soie sur facebook, soie sur iPhone. Voici une liste des jeux.

Le Figure~\ref{fig:} liste quelque partenaires : 

% TODO chugulu references.



\begin{itemize}
	\item LCL Open crémaillère \& 6:AM: Aménager entièrement son appart virtuel, et gagner réellement un an de loyer 
	\item Nokia Golden Goal \& Wunderman : Un advergame permettant de marquer plus de buts que l'équipe de France!
	\item Orange Snowballs \& Publicis Net : Un advergame pour i-Phone recréant une bataille de boules de neiges
	\item Air France \& La Chose : Un advergame s'inspirant du planisphère du magazine Air France
	\item Axe \& Buzzman : Un advergame prolongeant le film viral <<Canadairman>> pour la sortie d'AxeDry
	\item SFR \& EURO RSCG BETC 4D : Une réadaptation du Snake nokia 
	\item Renault \& Publicis Net : Contrer l'invasion de la dernière Scenic par des lapins vraiment très crétins
	\item Voyages SNCF \& Artdicted : Une narration ludo-pédagogique sur les gestes responsables du voyageur
	\item Prisma Press Group : Starbank, la bourse des people 
\end{itemize}

% subsubsection références (end)

\subsection{Advergame} % (fold)
\label{sub:advergame}

\subsubsection{Définitions} % (fold)
\label{ssub:définitions}

\begin{description}
	\item[Advergame] : jeu développé pour le compte d’un annonceur.
	\item[Casual gaming (ou jeu grand public)] : se dit des jeux vidéo s'adressant au plus grand nombre, femmes et seniors compris. Jeux présentant des gameplays simples et efficaces, compréhensibles en quelques secondes. Ex: PacMan, Tetris, AngryBirds, Doodle Jump, etc.
	\item[Social gaming] : jeux grand public (ou pas) conçus pour être distribués sur les réseaux sociaux (facebook en particulier) dont les gameplay inclus des fonctionnalités sociales et virales. Ex : Farmville, Diner Dash, Mafiawars, etc.
	\item[Serious game] : expérience ludique reprenant les codes du jeu vidéo, en vue de transmettre des acquis pédagogiques, responsables et/ou professionnalisants.
	\item[Freemium] : modèle dans lequel la découverte du jeu est gratuit (free) et le gameplay propose des biens virtuels payants (premium).
\end{description}
Le définition sur le site Wikipédia est : L'advergame ou jeu vidéo publicitaire est un jeu vidéo qui cherche uniquement à promouvoir l'image d'une marque. Le mot advergame est un néologisme peu utilisé en France qui vient de la contraction de advertising (publicité) et de game (jeu). Un jeu vidéo publicitaire n'est pas à proprement parlé un serious game (jeu sérieux). Les serious games sont en effet des applications utilisant les technologies, techniques et usages du jeu vidéo, mais dans un but non ludique: apprendre à faire la cuisine, apprendre à faire quelque chose (médecine, etc.), etc.
Le jeu vidéo publicitaire est un jeu à part entière avec son côté ludique et justement non sérieux. C'est ce qui attire les annonceurs: l'univers jeu vidéo.


\subsubsection{Le marché de l'advergaming} % (fold)
\label{ssub:le_marché_de_l_advergame}

Advergaming est un grand marché. Chiffre d’affaires mondial de 300 millions d’euros en 2010. Un taux de croissance annuel estimé à 47\% sur 2010 - 2015. Chiffre d’affaires 2015 estimé à 2 milliards d’euros. 40\% du top 100 des annonceurs français ont déjà eu recours à l’advergaming.



Au niveau de Fidélisation accrue des consommateurs, 61\% des joueurs déclarent avoir une meilleure opinion des produits présentés par un advergame. 50\% de la génération Y considèrent une marque plus crédible lorsqu'elle communique par le jeu.

Au niveau de Conversion prospect/client augmentée, 45\% des internautes satisfaits d’un advergame deviennent client de la marque.


% subsubsection le_marché_de_l_advergame (end)

% subsubsection définitions (end)

% subsection advergame (end)
\subsubsection{Expérience technique} % (fold)
\label{ssub:expérience_technique}

Chugulu game utilise plusieur techniques pour les jeux selon les besoins. Principalement, il existe 4 type de techniques. Ces 4 techniques sont les techniques moderne et populaire. 

\begin{description}
	\item[Ruby on Rails] Rails est un framework open-source optimisé pour le bonheur des développeurs, engendrant une productivité accrue. Il combine simplicité, efficacité et modularité, ce qui nous permet de répondre facilement et rapidement aux besoins de nos clients. Utilisé par un nombre croissant de sites internationaux majeurs (Twitter, Yellow Pages, Github, Justin.tv, etc.), Ruby on Rails fait partie à ce jour des technologie web les plus dynamiques et les plus intéressantes de par sa communauté.
	\item[Flash] est de très loin la solution la plus populaire pour diffuser du contenu interactif sur le web car il est présent, selon Adobe, sur 99,5\% des ordinateurs connectés à Internet en Europe. La popularité de son player fait d’Adobe Flash un outil de développement et d’animation 2D très utilisé pour les jeux sur le web et pour les applications internet riches.
	\item[L'iPhone] est aujourd'hui et depuis plusieurs années l'une des plateformes mobiles la plus plébicitées par les usagers. Récemment l'Ipad n'a fait que renforcer cette réalitée, c'est donc tout naturelement que Chugulu a consolider son experience dans cette technologie et plus largement dans les technologies mobiles.
	\item[Unity 3D] Particulièrement concerné par les technologies les plus innovantes, Chugulu s’est familiarisé avec Unity3D, un moteur de rendu 3D destiné au développement de jeu vidéo jouables directement en page web, mais aussi sur iPhone, iPad et les smartphones Android. Son moteur graphique << next gen >> permet de créer des jeux proposant des rendus exceptionnels, disponibles sur un maximum de plateforme.
\end{description}
% subsubsection expérience_technique (end)



\subsection{Introduciton au stage} % (fold)
Le mission de mon stage de fin d'étude est développement et suivi des applications iPhone de la société. L'objective contient ces quatre points suivants : 

\begin{itemize}
	\item Savoir travailler en équipe.
	\item Etre autonome sur une mission.
	\item Respecter les consignes du lead développer iPhone
	\item Faire preuve d'initiatives.
\end{itemize}

C



% subsection Introduciton au stage (end)
% chapter Présentation (end)



hn \chapter{Techniques informatiques} % (fold)
\label{cha:techniques_informatiques}

Dans ce chapitre, je vais parle des outils et des techniques utilisés pendant ce stage. Ces outils contient les outils utilisé dans l'équipe pour le but de mieux communiquer et mieux collaborer. Il y a des outils utilisé pour développement, y compris l'IDE, les outils pour débug et amélioration etc. Sauf-ci, je vais quand même présenter les frameworks, conception utilisé pour ce projet.

\section{Outils de Collaboration} % (fold)
\label{sec:outils_de_collaboration}

Pendant ce stage, j'ai touché plusieurs techniques nouvelles, appris beaucoup des façons à développement. Sauf ci, j'ai appris beaucoup des outils de collaborations, et apprendre la façon de travaille en équipe. Dans ce section, je vais parler des outils de collaborations

\subsection{Producteev et Redmine} % (fold)
\label{ssub:producteev_et_redmine}

La raison de mettre les 2 outils ensemble est non seulement les 2 outils se ressemblent, mais aussi, ils sont les plus importantes. Au début de stage, dans un première temps, ce que j'ai touché est le Producteev. Producteev est utilisé par Chugulu Games pendant certaines années. Cet été, ils ont décidé de changé à utiliser un autre outil qui s'appelle Redmine. 

\subsubsection{Producteev} % (fold)
\label{ssub:producteev}

% TODO Add the logo Producteev:

Producteev est un gestionnaire de tâches à la fois B2B et B2C dont l’ADN est collaboratif. D’une utilisation personnelle à un nombre de connectés illimités en entreprise, la solution fonctionne online et offline en interface avec les comptes e-mail, Facebook ou Google Calendar entre autres dans des versions web et mobiles. Il a ces fonctionnalités :
\begin{description}
	\item[Gestion de plusieurs équipes] Gérer plusieurs équipes grâce à Producteev en utilisant les fonctionnalités simples intégrées. Inviter les membres des équipes, ajustez les paramètres de confidentialité et bien plus encore pour votre liste de tâches.
	\item[Filtres et rapports] Générer facilement des rapports sur le projet et sur l'activité d'équipe. Par exemple, recevoir toutes les semaines un email d'aperçu des avancées d'équipe.
	\item[Emails récapitulatifs] Recevez chaque jour le rapport journalier/hebdomadaire, pour avoir un aperçu de l'équipe.
	\item[Tâches prioritaires]  L'algorithme utilisé pour la gestion des tâches prioritaires trie automatiquement les tâches et  suggère à faire ensuite
	\item[Synchroniser vos tâches avec Google Agenda] Synchronisez Producteev avec Google Agenda ou placer-y simplement une tâche à la fois. 
\end{description}

Sauf ces fonctionnalité, producteev est possible de synchroniser avec ses logiciel client sous Mac ou iPhone, avec Google. Et, il est possible de se communiquer en utilisant les mails. Envoyer et recevoir les nouvelles par les mails. 

% subsubsection producteev (end)

\subsubsection{Redmine} % (fold)
\label{ssub:redmine}

% TODO add the logo

Redmine est une application web Open Source de gestion complète de projet en mode web, développé en Ruby sur la base du framework Ruby on Rails.
Il a été créé par Jean-Philippe Lang. D'autres développeurs venant de la communauté des utilisateurs de Redmine contribuent depuis au projet. Il utilise la syntaxe Textile pour ses pages de wiki et de nombreux autres emplacements où l'utilisateur peut entrer du texte : 
\begin{itemize}
	\item descriptif de bogue ou de demande, ainsi que les commentaires associés
	\item forums
	\item nouvelles
	\item description
\end{itemize}

La fonctionnalité de Redmine comporte la liste suivante:
\begin{itemize}
	\item gestion multi-projets
	\item gestion fine des droits utilisateurs définis par des rôles
	\item gestion de groupes d'utilisateurs
	\item rapports de bogues (bugs), demandes d'évolutions
	\item Wiki multi-projets
	\item forums multi-projets
	\item news accessibles par RSS / ATOM
	\item notifications par courriel (mail)
	\item gestion de feuilles de route, GANTT, calendrier
	\item historique
	\item intégration avec divers suivis de versions : SVN, CVS, Mercurial, Git, Bazaar & Darcs
	\item identification possible via LDAP
	\item multilingue (25 langues disponibles pour la 0.7.0)
	\item support de plusieurs bases de données : MySQL, PostgreSQL ou SQLite
\end{itemize}

Par rapport au Producteev, redmine a des avantages evident. La plus importante est que redmine est gratuit. Il est possible d'installer redmine dans le site web interne d'entreprise. Il a un système de notification mieux que producteev. Redmine supporte beaucoup de personnalisation.

% subsubsection redmine (end)

% subsubsection producteev_et_redmine (end)

% section outils_de_collaboration (end)

\section{Outils de Développement} % (fold)
\label{sec:outils_de_développement}

% section outils_de_développement (end)

\section{Technique utilisé} % (fold)
\label{sec:technique_utilisé}

% section technique_utilisé (end)


% chapter techniques_informatiques (end)

\chapter{Développement d'un jeu} % (fold)
\label{cha:développement_d_un_jeu}

Dans ce chapitre, je vais parler du développement du jeu <<Playboy-spots>>. <<Playboy-spots>> est le premier jeu que j'ai participé dans ma vie. Au début jusqu'au produit finale. Pendant ces processus de développement, j'ai appris beaucoup de chose, concept etc. Non seulement que j'ai appris beaucoup, mais aussi, grâce à ce projet, j'ai commencé à réfléchir beaucoup, comparer les techniques utilisés, résumer les points négative et positive. Ces expérience vont m'aider beaucoup dans les projets futurs. Dans n'importe quelle entreprise.

\section{Processus} % (fold)
\label{sec:processus}

Un jeu est un logiciel spéciale. Au niveau de ressource, un jeu est plus difficile à développer par rapport aux logiciel. Parce que un jeu demande beaucoup de ressource par rapport aux logiciels. Je vais le parler dans section suivant. Au niveau de processus de développement. Un jeu est plus simple par rapport aux logiciel. Ici, je vais parler le processus de développement d'un jeu en général et en comparant avec le processus réel chez chugulu games. surtout, il faut savoir, ici, le processus est utilisé par les petites équipes. Pour les grandes entreprises comme <<EA>>, ils ont un autre processus plus complet, plus systématiques, mais va durer plus long, et leurs processus est utilisé pour créer des jeux gros. Pour chugulu games et billions des studio comme chugulu games, leurs caractéristiques commune sont petite, bouger simple, réactiver rapide, développer rapide. 

\subsection{Processus Général} % (fold)
\label{sub:processus_général}



% subsection processus_général (end)

% section processus (end)

% chapter développement_d_un_jeu (end)

\chapter{Conclusion}

Ce stage est une expérience très importante pour moi. C’est vraiment une expérience importante et magnifique pour moi de travailler à Chugulu. Grâce à vous, j’ai trouvé mon métier de prédilection.

J’ai développé le moteur de jeu du <<Playboy-spots>>. Dans le futur, il y aura certains projets qui sont basés sur ce moteur. S'il y a des projets similaires, je peux travailler dessus directement. Je n’ai pas besoin de relire le code, de familier avec les frameworks utilisées. 

Pendant ce stage, j’ai appris beaucoup en suivant les instructions de mon encadrant. Non seulement les techniques utilisées, les outils de collaborations, mais aussi les façons de communications, les façons de travailler dans l’équipe.  Les techniques ne sont pas difficiles à apprendre. L’équipe chugulu est une grande famille, ils sont tous très gentil pour moi un étranger. Grâce à eux, j’ai trouvé mon métier de prédilection. J’aime travailler dans l’industrie du jeu. Et j’ai de la chance de travailler dans une entreprise qui est le leader en France. Et dans le futur, je peux touché beaucoup de techniques nouvelles, avoir des expériences de jeux.



\annexes

Ici, je vais ajouter quelque capture du jeu <<Playboy-Spots>>.
\begin{figure}[htbp]
	\centering
		\includegraphics[height=3in]{Image/Capture1.png}
	\caption{Le jeu}
	\label{fig:Image_Capture1}
\end{figure}
\begin{figure}[htbp]
	\centering
		\includegraphics[height=3in]{Image/Capture2.png}
	\caption{Capture de la boutique}
	\label{fig:Image_Capture2}
\end{figure}
\begin{figure}[htbp]
	\centering
		\includegraphics[height=3in]{Image/Capture3.png}
	\caption{More games}
	\label{fig:Image_Capture3}
\end{figure}
\begin{figure}[htbp]
	\centering
		\includegraphics[height=3in]{Image/Capture4.png}
	\caption{Crédit}
	\label{fig:Image_Capture4}
\end{figure}


\end{document}

